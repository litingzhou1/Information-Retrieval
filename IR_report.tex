\documentclass{article}
\usepackage{graphicx}
\usepackage[margin=1in]{geometry}
\usepackage{amsmath,amssymb}

\begin{document}
\title{Report Information Retrieval}
\author{By Benno Kruit - Joost van Amersfoort - Otto Fabius \\ 10576223 - 10021248 - 5619858}
\maketitle

\section*{Introduction}
In this report, we detail experiments on several different models and parameter settings for retrieving documents matching a query. In these experiments we try to measure the effectiveness of each model and observe their strengths and weaknesses.  
In order to do this, an Information Retrieval system was developed for a subset of 262 documents from the CSIRO collection, as made available by the course. The system takes the set of documents in .txt format as input, stores information on these documents in an inverted index, and, for a given query, outputs a score for each document in a way that can be directly evaluated using the trec\_eval system. The IR system is implemented in Python.

First, although the emphasis is on the various models, we will briefly describe the preprocessing on and indexing of the documents, as done in our experiments. Next, we will describe the specific models implemented and determine the parameter settings to be varied. In the Results section, we will detail the effectiveness of our various models, evaluated using the trec\_eval system\footnote{http://trec.nist.gov/trec\_eval/} and a qrels file provided with the collection.

\section*{Preprocessing and Indexing}
In preprocessing, each document is split into lowercase tokens after removing punctuation. For each document, the frequency of each occurring token is stored in an inverted index. 

Additional optional preprocessing functions are included. One option is to remove stop words. Another option is to stem words (remove affixes) using the Python Natural Language ToolKit (NLTK)\footnote{http://nltk.org/} implementation of either the Porter stemmer or Lancaster stemmer. The Porter stemmer is less aggressive than the Lancaster stemmer, so we are expecting slightly different results.

The last option is to lemmatize words. Lemmatization is effectively a more elaborate way than stemming to group words with more or less the same meaning. Where stemming simply removes predefined endings of words when present, lemmatization converts words into their lemma, or base form, through attempting to match words to a dictionary. This way, words that differ distinctly morphologically from their base form (such as 'worse' does from 'bad') can also be lemmatized. This is, however, computationally more costly than stemming. For the optional preprocessing functions, again implementations by NLTK are used.

\section*{Models}
In this section we detail the models we implemented to evaluate on the used subset of the CSIRO collection. We implemented TF-IDF and BM-25, as these are relatively simple, much-used retrieval models and thus produce good baseline scores. Then, as an experiment, we implemented Query Expansion as part of preprocessing the query, Parsimonious Language Models, and Cosine Similarity based on term frequency.
\subsection*{TF-IDF}
TF-IDF scoring is based on the term frequency in a given document compared to the total term frequency, and is designed to return high scores if the term is frequent in a document (hence Term Frequency and infrequent in the whole collection (hence Inverse Document Frequency). We use logarithmically scaled term frequency, resulting in the following formula for the score of a document $d$, given a query $q$:
\begin{align*}
Score = (1+log(TF))\cdot IDF
\end{align*}
With $TF$ being the frequency of $q$ in $d$
\begin{align*}
IDF = log(\frac{N_{documents, total}}{N_{dcuments, q}})
\end{align*}
\subsection*{BM25}
Similar to TF-IDF, BM25 is a bag-of-words retrieval model. With $IDF$ computed as for TF-IDF, the score is computed as:
\begin{align*}
Score(d,q) = IDF(q) * \frac{(k+1)*TF}{k*(1-b)+b\frac{length_d}{length_{mean}}+TF}
\end{align*}
With $k$ and $b$ parameters set to k = 1.5, b = 0.75, as suggested by \cite{robertson1995okapi}.

\subsection*{Parsimonious Language Model}

Parsimonious Language modeling is based on iteratively re-estimating term probability $P(t|D)$ using Expectation Maximization:

\begin{align*}
e_t &= tf(t,D)\frac{\lambda P(t|D)}{(1-\lambda)P(t|C)+\lambda P(t|D)} \\
P(t|D) &= \frac{e_t}{\sum_t e_t}
\end{align*}
In our experiments, we set parameter $\lambda = 0.5$ and perform ten of these iterative steps.\\ \\
For retrieving the ordered list of relevant documents for query $R$, the cross-entropy is calculated using $P(t|D)$ and $P(t|R)$, which are acquired through EM:
\begin{align*}
H(R,D) = -\sum_t P(t|R)\cdot log\big( \ (1-\lambda)P(t|C)+\lambda P(t|D) \ \big)
\end{align*}

\subsection*{Word Vectors}

In this retrieval model, we define the score of a document and query as the cosine similarity between their vector representations. Each element of a vector representation is the frequency of the corresponding word in the document or query. A vector thus has the length of the size of the preprocessed vocabulary. Effectively, this retrieval system is therefore a frequency count with a different structure.

The cosine similarity between vectors $A$ and $B$ is the cosine of their angle $\theta$, and is defined as follows:

\begin{align*}
\text{similarity}(A,B) = \cos(\theta) = {A \cdot B \over \|A\| \|B\|} 
\end{align*}


\subsection*{Query Expansion}
In query expansion, an attempt is made to expand the query with words that are similar to the query according to some measure. In our implementation, first the top $n$ documents for the query are selected using the defined retrieval model. Next, the $m$ most frequent tokens in the returned documents are selected to add to the original query for re-evaluation of the scores. The frequency of a token in the relevant documents can be normalized for the length of each of the top $n$ documents. We therefore have two measures:

\begin{align*}
F_{abs}(term) = \sum_{doc=1}^{n}freq(term,doc)
\end{align*}
and
\begin{align*}
F_{rel}(term) = \sum_{doc=1}^{n}\frac{freq(term,doc)}{length(doc)}
\end{align*}

In our results, we use relatively small $n = 20$ and $m = 10$ because we have a small collection. Unfortunately, Query Expansion did not give any performance improvement and often degraded performance. Therefore, we do not include it in the result section. An idea to make this perform better is to use it on a much larger dataset and use a smarter way to select new words from the top document.

\section*{Results}

Please see the included README.txt file for instructions on using the program and replicating our results.

\subsection*{Preprocessing}

First, we compared results for different combinations of preprocessing techniques to ascertain which combinations were effective for our document collection and queries, using TF-IDF. An overview is given in Tables 1 and 2. Clearly, the influence of preprocessing on our results is minimal, at least for TF-IDF.

\begin{table}[h]
\begin{tabular}{|l|c|r|}
\hline
 & MAP & R-precision \\ \hline
Porter stemming & 0.6454 & 0.6622 \\ \hline
Lancaster stemming & 0.6454 & 0.6622 \\ \hline
Porter stemming and stopword removal & 0.6454 & 0.6622 \\ \hline
Lemmatization and stopword removal & 0.6454 & 0.6622 \\ \hline
\end{tabular}
\caption{MAP and R-precision for TF-IDF using four different preprocessing settings. Query: 'sustainable ecosystems'}
\end{table}
\begin{table}[h]
\begin{tabular}{|l|c|r|}
\hline
 & MAP & R-precision \\ \hline
Porter stemming & 0.8191 & 0.6491 \\ \hline
Lancaster stemming & 0.8191 & 0.6491 \\ \hline
Porter stemming and stopword removal & 0.8165 & 0.6316 \\ \hline
Lemmatization and stopword removal & 0.8191 & 0.6491 \\ \hline
\end{tabular}
\caption{MAP and R-precision for TF-IDF using four different preprocessing settings. Query: 'air guitar textile sensors'} 
\end{table} 

\subsection*{Comparison of Retrieval Models}

Next, we compared our different retrieval models for the same queries. For BM-25, we used parameter settings k = 1.5 and b = 0.75, which are the suggested values by \cite{robertson1995okapi}. In our implementation of Parsimonious Language Modeling, we used 10 iterations of the EM algorithm, with $\lambda$ set to 0.1 as suggested by \cite{hiemstra2004parsimonious}. More iterations did not improve the result.  MAP and R-precision were calculated by trec\_eval on the basis of our output and the qrels file, and are reported in Table 3 and Table 4, respectively.

\begin{table}[h]
\caption{Mean Average Precision for our different retrieval models, for both queries}
\begin{tabular}{|l|c|c|}
\hline
Query & 'sustainable ecosystems' & 'air guitar textile sensors' \\ \hline
TF-IDF & 0.6454 & 0.8191 \\ \hline
BM-25 & 0.7326 & 0.7753 \\ \hline
PLM & 0.5233 & 0.6679 \\ \hline
Word Vector & 0.7979	 & 0.7297 \\ \hline
\end{tabular}
\end{table}

\begin{table}[h]
\caption{R-Precision for our different retrieval models, for both queries}
\begin{tabular}{|l|c|c|}
\hline
Query & 'sustainable ecosystems' & 'air guitar textile sensors' \\ \hline
TF-IDF & 0.6622 & 0.6491 \\ \hline
BM-25 & 0.7027 & 0.5614 \\ \hline
PLM & 0.6081 & 0.5439 \\ \hline
Word Vector & 0.7297 & 0.7719 \\ \hline
\end{tabular}
\end{table}

We also tested the effectiveness of query expansion on TF-IDF, reported in table 5.

\begin{table}[h]
\caption{MAP and R-Precision for query expansion with TF-IDF, averaged over both queries}
\begin{tabular}{|l|c|c|}
\hline
 & MAP & R-precision \\ \hline
No QE & 0.7322 & 0.6556 \\ \hline
Absolute QE & 0.7191 & 0.7170 \\ \hline
Relative QE & 0.7191 & 0.7170 \\ \hline
\end{tabular}
\end{table}

\section*{Discussion}

\subsection*{Preprocessing}

As mentioned in the Results, our different methods for preprocessing did not have a significant impact on MAP and R-precision scores. 

First of all, the lemmatization and either stemming algorithm used did not make any difference for either query. This was not surprising considering the queries are stemmed in the same way: "sustain ecosytem" and "air guitar textil sensor". 

Removing stopwords also had little to no effect (see Table 1 and Table 2). Of course, our queries did not contain any stopwords, so the only influence removing stopwords might have is on the length of documents, which is used in calculating the score with TF-IDF. 

\subsection*{Comparing Models}

As can be seen from Table 3 and Table 4, the results using a PLM are worse than TF-IDF and BM-25 for both queries. Which retrieval method was best, depended on the query.

Somewhat disappointingly, PLM did not achieve a great result. It is not entirely certain what the reason is, but we suspect that the size of the dataset was not sufficient for a proper Language Model. Our collection is a subset of the true collection, but it does contain all the relevant documents for each query. This means that there are a relatively large amount of relevant documents, which favors BM25 and TFIDF. As they are more easily "confused" by large datasets, while PLM could thrive in a situation with less relevant documents.

The word vector model achieved remarkably high scores for both queries, highlighting that a simple frequency count can already prove effective.

The query expansion worsened the results slightly for TF-IDF. One possible reason is that the data for the query expansion was not only the same data as we retrieve from (an external collection is preferable), but also that the used collection was quite small. this limited the number of relevant documents to select expansion terms from, which resulted in mostly unrelated expansion terms.

\section*{Conclusion}

Our results suggest that the chosen model is more important than the preprocessing method of the document collection. For our queries and document collection, BM-25 and TF-IDF and a word vector model for frequency counts outperformed PLM. The successfulness of a model seems to depend largely on the query and possibly also on the type and size document collection. Therefore, we can not make solid conclusions without more testing and different datasets. We can, however, conclude that simple algorithms (TF-IDF) that are trivial to implement, achieve acceptable scores even for a small dataset and are not easily improved upon.

\bibliographystyle{amsplain}
\bibliography{ir}
\end{document}
